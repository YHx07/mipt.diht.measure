\begin{task}{5}
Являются ли следующие системы полукольцом, кольцом, алгеброй:
\begin{enumerate}
    \item[(a)]  Полуинтервалы: $S = \{[\alpha; \beta) |~\alpha, \beta \in R\}$;
    \item[(b)] Все конечные подмножетсва натуральных чисел;
    \item[(c)] Все измеримые по Жордану подмножества отрезка [0, 1];
    \item[(d)] Все открытые множества на прямой.
\end{enumerate}
\end{task}
\begin{solution}
Любая $\sigma$-алгебра является алгеброй, любая алгебра является кольцом, любое кольцо является полукольцом.
\begin{enumerate}
    \item[(a)] Докажем что не является кольцом. 
    
    Возьмем полуинтервалы $A=[0,3)$ и $B=[1,2)$. $A \triangle B=[0,1) \cap [2,3) \notin S$. Хотя симметрическая разность должна принадлежать кольцу.
    
    Докажем, что является полукольцом.
    \begin{itemize}
        \item $\varnothing \in S$.
        
        Возьмем $\alpha = \beta$, $[\alpha, \beta) = \varnothing$, то есть $\varnothing \in S$.
        \item если $A, B \in S$, то $A \cap B \in S$.
        
        Пересечение двух полуинтервалов --- полуинтервал. Значит пересечение принадлежит полуинтервалу.
        \item если $A, A_1 \in S$ и $A_1 \subset A$, то существуют конечное число множеств $A_2, A_3,\dots, A_n \in S$ таких, что $A = A_1 \sqcup A_2 \sqcup \dots \sqcup A_n$.
        Очевидно хватит двух $B_1, B_2 \in S$ (возможно пустых), чтобы дополнить $A_1$ до $A$.
    \end{itemize}
    
    \item[(b)] Докажем что не является алгеброй.
    
    Назовем множество всех конечных подмножеств натуральных чисел --- $S$.
    Возьмем $A \subset S$. $\overline{A}$ не будет конечным, значит не будет лежать в  $S$. Значит $S$ не образует алгебру.
    
    Очевидно, что является кольцом.
    \begin{itemize}
        \item $S$ непусто;

        \item если $A, B \in S$, то $A \cap B \in S$;
        
        \item если $A, B \in S$ то $A \triangle B \in S$.
    \end{itemize}
    
    \item[(c)] Является алгеброй.
    \begin{itemize}
        \item $\varnothing \in S$;

        Пустое множество измеримо по Жордану.
        \item если $A \in S$, то $\overline{A} \in S$;
        
        По свойству внешней и внутренней мер.
        \item если $A, B \in S$, то $A \cup B \in S$;
        
        Из определения меры.
    \end{itemize}
    
    \item[(d)] Не является даже полукольцом. 
    
    Назовем множество всех открытых множеств на прямой --- $S$.
    Возьмем $A \in S$ и $B \in S$ такое, что $B \subsetneq A$ и левые концы $A$ и $B$ совпадают. Тогда $A \cap B$ -- это полуинтервал, а полуинтервал не является открытым множеством. Значит $S$ не полукольцо.
\end{enumerate}
\end{solution}
