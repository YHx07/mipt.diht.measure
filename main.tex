\documentclass[12pt,a4paper]{article}
\usepackage[l2tabu,orthodox]{nag}

% Пожалуйста, не меняйте указанные ниже установки полей в документе
\usepackage[left=10mm,right=15mm, top=15mm,bottom=15mm,bindingoffset=0cm]{geometry}

\usepackage{indentfirst}

\usepackage[shortlabels]{enumitem}

\usepackage{ccaption}
\captiondelim{. }
\usepackage{mathtools}
\usepackage{amssymb,amsmath,amsthm}

\usepackage{fontspec}
%\usepackage{unicode-math}
\usepackage{epigraph} %Эпиграф
\renewcommand{\epigraphrule}{0pt} 

\setmainfont[Ligatures=TeX]{STIX}
\newfontfamily{\cyrillicfont}[Ligatures=TeX]{STIX}
\setmonofont{Fira Mono}
\newfontfamily{\cyrillicfonttt}{Fira Mono}

\usepackage{polyglossia}
\setdefaultlanguage{russian}
\setotherlanguage{english}

\usepackage{graphicx}
\graphicspath{{img/}}
\DeclareGraphicsExtensions{.pdf,.png,.jpg}


\usepackage{color}
\definecolor{darkblue}{rgb}{0,0,.75}
\definecolor{darkred}{rgb}{.7,0,0}
\definecolor{darkgreen}{rgb}{0,.7,0}

\usepackage[normalem]{ulem}
\usepackage[textwidth=4cm,textsize=tiny]{todonotes}
\newcommand{\hl}[1]{{\textcolor{red}{#1}}}  %можно выделить текст красным цветом
\usepackage[pverb-linebreak=no]{examplep}
\usepackage{amsmath}

\usepackage[
    draft = false,
    unicode = true,
    colorlinks = true,
    allcolors = blue,
    hyperfootnotes = true
]{hyperref}


\theoremstyle{plain}
\newtheorem{theorem}{Теорема}
\newtheorem{lemma}{Лемма}
\newtheorem{proposition}{Утверждение}
\newtheorem{corollary}{Следствие}
\theoremstyle{definition}
\newtheorem{definition}{Определение}
\newtheorem{notation}{Обозначение}
\newtheorem{example}{Пример}
     
\newenvironment{task}[1]
    {\addtocounter{subsubsection}{1}
    \addcontentsline{toc}{subsubsection}{Задача #1}
    \par\noindent\textbf{Задача \href{https://vk.com/doc3656763_485132047?hash=4c348f38c8d21a7c4c&dl=02e7c731e921487b47}{#1}. }}
    {\smallskip}
\newenvironment{solution}
    {\par\noindent\textbf{Решение. }}
    {\bigskip}
    
\DeclareMathOperator*\lowlim{\underline{lim}}
\DeclareMathOperator*\uplim{\overline{lim}}

\title{Подготовка к коллоквиуму по теории меры}
\author{ФИВТ, 2 курс}

\begin{document}
\tracingall
\maketitle
\epigraph{\textit{Одного счетного пересечения объединения и последний шаг кокаином остался.}}{}
\epigraph{\textit{Я назначаю кольцо с ее дички то есть я входил всегда везде.}}{}
\epigraph{\textit{Дети не элемент никакого кольца не плод моего конца.}}{}
\epigraph{\textit{Когда мы идем считать мы считаем.}}{\textit{VK::ЭРЛРЭ}}
\newpage
\tableofcontents


%\newpage
%\section{Системы множеств.}
%\begin{task}{1}
Доказать, что
\begin{equation*}
    \overline{\lim_n}A_n =\displaystyle{\bigcap_n} \Big(\bigcup_{k \ge n} A_k\Big) \;\;\;\;\; \underline{\lim}_n A_n =\displaystyle{\bigcup_n} \Big(\bigcap_{k \ge n} A_k\Big).
\end{equation*}
\end{task}
\begin{solution}
%Сначала разберемся с верхним пределом. Все элементы, которые пренадлежат бесконечному числу множеств из $A_n$ будут попадать в пересечение, так как всегда для любого $n$ найдется $k$, такое, что $x \in A_k$ по определению бесконечной подпоследовательности. Теперь заметим, что все элементы, которые принадлежат пересечению, пренадлежат бесконечному числу множеств из $A_n$, так как если бы они принадлежали конечному числу множеств, начиная с некоторого номера, они бы перестали попадать в пересечение. Значит правая часть эквивалента определению верхнего предела. С нижним пределом эквивалентные рассуждения.

Верхний предел последовательности множеств $A_n$ состоит из тех и только тех элементов $x$, каждый из которых принадлежит бесконечному числу множеств последовательности $A_n$.

Нижний предел последовательности множеств $A_n$ состоит из тех и только тех элементов $x$, каждый из которых принадлежит всем множествам последовательности $A_n$, за исключением, быть может, конечного числа. 

Верхний предел.

Пусть $x \in \underset{n}{\bigcap} \left( \underset{k \geq n}{\bigcup} A_k \right)$, тогда $\forall n ~ x \in \underset{k \geq n}{\bigcup} A_k$, откуда следует, что $\exists \lbrace n_k \rbrace : \forall k ~ x \in a_{n_k}$, значит $x \in \underset{n \to \infty}{\varlimsup} A_n$.

Нижний предел.

Пусть $x \in \underset{n}{\bigcup} \left( \underset{k \geq n}{\bigcap} A_k \right)$, тогда $\exists N : x \in \underset{k \geq N}{\bigcap} A_k$, значит $x \in \underset{n \to \infty}{\varliminf} A_n$.

Получили, что $\underset{n}{\bigcup} \left( \underset{k \geq n}{\bigcap} A_k \right) \subseteq \underset{n \to \infty}{\varliminf} A_n$ и $\underset{n}{\bigcap} \left( \underset{k \geq n}{\bigcup} A_k \right) \subseteq \underset{n \to \infty}{\varlimsup} A_n$.

Обратные включения доказываются тривиально. 
\end{solution}
%\begin{task}{2}
Доказать, что если последовательность множеств $\{A_n\}$ монотонна, то
\begin{equation*}
    \overline{\lim_n}A_n  = \underline{\lim}_n A_n.
\end{equation*}
При этом $\lim_n A_n = \bigcup_n A_n$, если $A_n$ возрастают и $\lim_n A_n = \bigcap_n A_n$, если $A_n$ убывают.
\end{task}
\begin{solution}
%Доказательство проводится с помощью представления верхнего и нижнего пределов как в предыдущей задаче.
Рассмотрим два варианта:
\begin{enumerate}
    \item $\lbrace A_n \rbrace$ монотонно возрастает, т.е. $\forall n \hookrightarrow A_n \subset A_{n+1}$.
    
    $ \underset{n \to \infty}{\varliminf} A_n = \underset{n}{\bigcup} \left( \underset{k \geq n}{\bigcap} A_k \right) = \underset{n}{\bigcup} \left(A_n\right) = \underset{n}{\bigcup} A_n$, так как из монотонности следует, что $\forall ~ n \underset{k \geq n}{\bigcap} A_k = A_n$.
    
    $\underset{n \to \infty}{\varlimsup} A_n = \underset{n}{\bigcap} \left( \underset{ k \geq n}{\bigcup} A_k \right) = \underset{n}{\bigcap} \left(\underset{k}{\bigcup} A_k\right) = \underset{n}{\bigcup} A_n$, так как из монотонности следует, что $\forall n ~ \underset{k \geq n}{\bigcup} A_k = \underset{p}{\bigcup} A_p$.
    
    \item $\lbrace A_n \rbrace$ монотонно убывает, т.е. $\forall ~ n \hookrightarrow A_{n+1} \subset A_n$.
    
     $ \underset{n \to \infty}{\varliminf} A_n = \underset{n}{\bigcup} \left( \underset{k \geq n}{\bigcap} A_k \right) = \underset{n}{\bigcup} \left(\underset{k}{\bigcap}A_k\right) = \underset{n}{\bigcap} A_n$, так как из монотонности следует, что $\forall n ~ \underset{k \geq n}{\bigcap} A_k = \underset{p}{\bigcap} A_p$.
    
    $\underset{n \to \infty}{\varlimsup} A_n = \underset{n}{\bigcap} \left( \underset{ k \geq n}{\bigcup} A_k \right) = \underset{n}{\bigcap} \left(A_n\right) = \underset{n}{\bigcap} A_n$, так как из монотонности следует, что $\forall n ~ \underset{k \geq n}{\bigcup} A_k = A_n$.
\end{enumerate}

Из вышеописанных равенств следует то, что нам надо в задаче.
\end{solution}
%\begin{task}{3}
Привести пример последовательности $A_1$, $A_2$, ..., что $\overline{\lim_n}A_n  \neq \underline{\lim}_n A_n.$ Доказать, что
\begin{equation*}
    \overline{\overline{\lim_n}A_n}  = \underline{\lim}_n \overline{A_n}.
\end{equation*}
\end{task}
\begin{solution}
Пример: $A_{2k} = \{1\}$, $A_{2k + 1} = \varnothing$.
\begin{equation*}
    \overline{\varlimsup_{n \to \infty} A_n} = \overline{ \underset{n}{\bigcap} \left( \underset{k \geq n}{\bigcup} A_k \right)} = \underset{n}{\bigcup} \left( \underset{k \geq n}{\bigcap} \overline{A}_k \right) = \varliminf_{n \to \infty} \overline{A}_n
\end{equation*}
\end{solution}

%\begin{task}{4}
Пусть $f: A \rightarrow B$ --- отображение множеств, $\mathfrak{A}$ -- система подмножетсв множества $A$, $\mathfrak{B}$ -- система подмножеств множества $B$. Положим
\begin{align*}
    f(\mathfrak{A}) &= \{f(X) \subset B: X \in \mathfrak{A}\}\\
    f^{-1}(\mathfrak{B}) &= \{f^{-1}(Y) \subset A: Y \in \mathfrak{B}\}.\\
\end{align*}
\begin{enumerate}[(a)]
    \item Показать, что $f(\mathfrak{A})$, вообще говоря, не обязано быть кольцом, если $\mathfrak{A}$ -- кольцо.
    \item Доказать, что если $\mathfrak{B}$ -- кольцо ($\sigma$-алгебра), то $f^{-1}(\mathfrak{B})$ -- кольцо ($\sigma$-алгебра).
\end{enumerate}

\end{task}
\begin{solution}
\begin{enumerate}[(a)]
    \item $A = \{1, 2, 3, 4\}$, $B = \{a, b, c\}$  и $\mathfrak{A} = \{\varnothing, \{1, 2\}, \{3, 4\}, A\}$, очевидно $\mathfrak{A}$ -- кольцо. Пусть $f(1) = a$, $f(2) = f(3) = b$, $f(4) = c$. Тогда $f(\mathfrak{A}) = \{\varnothing, \{a, b\}, \{b, c\}, B\}$ -- не кольцо, потому что $\{a, b\} \cap \{b, c\} = \{b\} \notin f(\mathfrak{A})$.
    
    \item (не уверен) Пусть $\mathfrak{B}$ -- кольцо ($\sigma$-алгебра).
    \begin{enumerate}[1)]
        \item $\varnothing \in f^{-1}(\mathfrak{B})$, так как $\varnothing = f^{-1}(\varnothing)$, а $\varnothing \in \mathfrak{B}$.
        
         \item
        $A, B \in  f^{-1}(\mathfrak{B}) \Rightarrow \exists \Phi,\Psi :  A = f^{-1}(\Phi), B = f^{-1}(\Psi)$.
        \begin{itemize}
            \item $A \cap B \in  f^{-1} ( \mathfrak{B} )$, так как $f^{-1} ( \Phi \cap \Psi ) = f^{-1} ( \Phi ) \cap f^{-1} ( \Psi ) = A \cap B$, а $\Phi \cap \Psi \in \mathfrak{B}$.
            \item $A \Delta B \in  f^{-1} ( \mathfrak{B} )$, так как $f^{-1} ( \Phi \Delta \Psi ) = f^{-1} ( \Phi ) \Delta f^{-1} ( \Psi ) = A \Delta B$, а $\Phi \Delta \Psi \in \mathfrak{B}$.
        \end{itemize}
        
        \item
        Пусть $E$ -- единица в $\mathfrak{B}$. Тогда $\forall X \in \mathfrak{B} ~ X \subseteq E$. 
        
        Тогда из того, что $\forall A, B \in \mathfrak{B} : A \subseteq B$ выполнено $f^{-1}(A) \subseteq f^{-1} (B)$, следует, что $f^{-1} (E)$ -- единица в $f^{-1} ( \mathfrak{B} )$.
        
        \item
        Пусть $A_1 \ldots A_n \ldots \in f^{-1} ( \mathfrak{B} )$, тогда $\exists ~ B_1 \ldots B_n \ldots ~ \in \mathfrak{B} : ~ \forall n ~ A_n = f^{-1} (B_n)$. Тогда $\underset{n}{\bigcup}A_n \in f^{-1} ( \mathfrak{B} )$, так как $f^{-1} ( \underset{n}{\bigcup} B_n ) = \underset{n}{\bigcup} f^{-1} (B_n) = \underset{n}{\bigcup} A_n$, а $\underset{n}{\bigcup} B_n \in \mathfrak{B}$
        
        Из первых двух равенств следует, что $f^{-1} ( \mathfrak{B} )$ -- кольцо, если $\mathfrak{B}$ -- кольцо или $\sigma$-алгебра, а из последних двух следует, что $f^{-1}(\mathfrak{B})$ -- $\sigma$-алгебра, только если $\mathfrak{B}$ -- $\sigma$-алгебра.
    \end{enumerate}
\end{enumerate}
\end{solution}

%\begin{task}{5}
Являются ли следующие системы полукольцом, кольцом, алгеброй:
\begin{enumerate}
    \item[(a)]  Полуинтервалы: $S = \{[\alpha; \beta) |~\alpha, \beta \in R\}$;
    \item[(b)] Все конечные подмножетсва натуральных чисел;
    \item[(c)] Все измеримые по Жордану подмножества отрезка [0, 1];
    \item[(d)] Все открытые множества на прямой.
\end{enumerate}
\end{task}
\begin{solution}
Любая $\sigma$-алгебра является алгеброй, любая алгебра является кольцом, любое кольцо является полукольцом.
\begin{enumerate}
    \item[(a)] Докажем что не является кольцом. 
    
    Возьмем полуинтервалы $A=[0,3)$ и $B=[1,2)$. $A \triangle B=[0,1) \cap [2,3) \notin S$. Хотя симметрическая разность должна принадлежать кольцу.
    
    Докажем, что является полукольцом.
    \begin{itemize}
        \item $\varnothing \in S$.
        
        Возьмем $\alpha = \beta$, $[\alpha, \beta) = \varnothing$, то есть $\varnothing \in S$.
        \item если $A, B \in S$, то $A \cap B \in S$.
        
        Пересечение двух полуинтервалов --- полуинтервал. Значит пересечение принадлежит полуинтервалу.
        \item если $A, A_1 \in S$ и $A_1 \subset A$, то существуют конечное число множеств $A_2, A_3,\dots, A_n \in S$ таких, что $A = A_1 \sqcup A_2 \sqcup \dots \sqcup A_n$.
        Очевидно хватит двух $B_1, B_2 \in S$ (возможно пустых), чтобы дополнить $A_1$ до $A$.
    \end{itemize}
    
    \item[(b)] Докажем что не является алгеброй.
    
    Назовем множество всех конечных подмножеств натуральных чисел --- $S$.
    Возьмем $A \subset S$. $\overline{A}$ не будет конечным, значит не будет лежать в  $S$. Значит $S$ не образует алгебру.
    
    Очевидно, что является кольцом.
    \begin{itemize}
        \item $S$ непусто;

        \item если $A, B \in S$, то $A \cap B \in S$;
        
        \item если $A, B \in S$ то $A \triangle B \in S$.
    \end{itemize}
    
    \item[(c)] Является алгеброй.
    \begin{itemize}
        \item $\varnothing \in S$;

        Пустое множество измеримо по Жордану.
        \item если $A \in S$, то $\overline{A} \in S$;
        
        По свойству внешней и внутренней мер.
        \item если $A, B \in S$, то $A \cup B \in S$;
        
        Из определения меры.
    \end{itemize}
    
    \item[(d)] Не является даже полукольцом. 
    
    Назовем множество всех открытых множеств на прямой --- $S$.
    Возьмем $A \in S$ и $B \in S$ такое, что $B \subsetneq A$ и левые концы $A$ и $B$ совпадают. Тогда $A \cap B$ -- это полуинтервал, а полуинтервал не является открытым множеством. Значит $S$ не полукольцо.
\end{enumerate}
\end{solution}

%\begin{task}{6}
Доказать, что набор множеств, замкнутый относительно операций
\begin{enumerate}[a)]
    \item $\cap$ и $\cup$,
    \item $\cap$ и $\backslash$ может не быть кольцом.
\end{enumerate}

\end{task}
\begin{solution}
    \begin{enumerate}[a)]
        \item
        $S = \lbrace \varnothing, \lbrace 1 \rbrace, \lbrace 1, 2 \rbrace \rbrace $ замкнут относительно $\cup$ и $\cap$, но кольцом не является, потому что $\{1,2\}\triangle\{1\} = \{2\} \notin S$.
        
        \item
        $S = \lbrace \varnothing, \lbrace 1 \rbrace, \lbrace 2 \rbrace \rbrace$ замкнут относительно $\cap$ и $\setminus$, но кольцом не является, потому что $\{1\} \triangle \{2\} = \{1, 2\} \notin S$.
    \end{enumerate}
\end{solution}
%\begin{task}{7}
Пусть $\mathfrak{B_1}$ и $\mathfrak{B_2}$ --- две $\sigma$-алгебры подмножеств пространства $\Omega$. Являются ли $\sigma$-алгебрами классы множеств: 
\begin{enumerate}[1)]
    \item $\mathfrak{B_1} \cap \mathfrak{B_2}$;
    \item $\mathfrak{B_1} \cup \mathfrak{B_2}$;
    \item $\mathfrak{B_1} \setminus \mathfrak{B_2}$;
    \item $\mathfrak{B_1} \triangle \mathfrak{B_2}$.
\end{enumerate}

\end{task}
\begin{solution}
\begin{enumerate}[1)]
    \item $\mathfrak{B} = \mathfrak{B}_1 \cap \mathfrak{B}_2$ -- $\sigma$-аглебра:
    \begin{enumerate}[a)]
        \item
        $\varnothing \in \mathfrak{B}_1, ~ \varnothing \in \mathfrak{B}_2 \Rightarrow \varnothing \in \mathfrak{B}$.
        
        \item
        $X,~Y \in \mathfrak{B} \Rightarrow X,~Y \in \mathfrak{B}_1; ~ X,~Y \in \mathfrak{B}_2 \Rightarrow X \cap Y \in \mathfrak{B};~ X \Delta Y \in \mathfrak{B}$, так как $X \cap Y \in \mathfrak{B}_1,~ X \Delta Y \in \mathfrak{B}_1$ и $X \cap Y \in \mathfrak{B}_2,~ X \Delta Y \in \mathfrak{B}_2$.
        
        \item
        $A_1, \ldots, A_n, \ldots \in \mathfrak{B} \Rightarrow A_1, \ldots, A_n, \ldots \in \mathfrak{B}_1; A_1, \ldots, A_n, \ldots \in \mathfrak{B}_2 \Rightarrow \underset{n}{\bigcap} A_n \in \mathfrak{B}$, так как $\underset{n}{\bigcap} A_n \in \mathfrak{B}_1$ и $\underset{n}{\bigcap} A_n \in \mathfrak{B}_2$.
        
        \item
        $\Omega \in \mathfrak{B} \Rightarrow \Omega \in \mathfrak{B}_1;~\Omega \in \mathfrak{B}_2 \Rightarrow \Omega$ -- единица в $\mathfrak{B}$.
    \end{enumerate}
    
    \item
    $\mathfrak{B_1} = \{\mathbb{R}, \varnothing, (-\infty; 1), [1; +\infty)\}$
    
    $\mathfrak{B_2} = \{\mathbb{R}, \varnothing, (-\infty; 2), [2; +\infty)\}$
    
    $\mathfrak{B_1} \cup \mathfrak{B_2} = \{\mathbb{R}, \varnothing, (-\infty; 1), [1; +\infty), (-\infty; 2), [2; +\infty)\}$
    
    Пересечение двух элементов из объединения $(-\infty; 2) \cap [1; +\infty) = [1, 2) \notin~\mathfrak{B_1}~\cup~\mathfrak{B_2}$, значит объединение не является даже кольцом, поэтому $\mathfrak{B_1} \cup \mathfrak{B_2}$ --- не $\sigma$-алгебра.
    
    \item
    $\mathfrak{B_1} = \{\mathbb{R}, \varnothing, (-\infty; 1), [1; +\infty)\}$
    
    $\mathfrak{B_2} = \{\mathbb{R}, \varnothing, (-\infty; 2), [2; +\infty)\}$
    
    $\mathfrak{B} = \mathfrak{B}_1 \setminus \mathfrak{B}_2 = \{(-\infty; 1);[1;+\infty)\} $ -- не $\sigma$-алгебра, так как $\varnothing \notin \mathfrak{B}$.
    
    \item
    $\mathfrak{B_1} = \{\mathbb{R}, \varnothing, (-\infty; 1), [1; +\infty)\}$
    
    $\mathfrak{B_2} = \{\mathbb{R}, \varnothing, (-\infty; 2), [2; +\infty)\}$
    
    $\mathfrak{B} = \mathfrak{B}_1 \triangle \mathfrak{B}_2 = \{(-\infty; 1),[1;+\infty),(-\infty; 2), [2; +\infty)\} $ -- не $\sigma$-алгебра, так как $\varnothing \notin \mathfrak{B}$.
\end{enumerate}
\end{solution}
%\begin{task}{8}
Доказать, что всякая конечная $\sigma$-алгебра подмножеств пространства $\Omega$ пораждается некоторым конечным разбиением $\Omega$. Доказать, что мощность всякой конечной $\sigma$-агебры является степенью двойки.
\end{task}

\begin{solution}
Пусть $A$ -- наша $\sigma$-аглебра с единицей $\Omega$. Рассмотрим разбиение единицы: $\Omega = \underset{k=1}{\overset{n}{\bigsqcup}} \Omega_k$, обладающее следующими свойством:
\begin{enumerate}
    \item Любое множество из разбиения обязано быть только подмножеством каких-то элементов $A$, то есть нет такого множества $X \in A$ и индекса $k$, что $X \cap \Omega_k \neq \Omega_k$, при условии, что $X \neq \varnothing$
\end{enumerate}
Такое разбиение существует, так как:
\begin{enumerate}
    \item В силу того, что $A$ -- полукольцо, то существует конечное разбиение $\Omega$.
    
    \item Если текущее разбиение не удовлетворяет условиям, то мы можем каждый элемент разбить еще так, чтобы новое разбиение стало удовлетворять условию (разбивать будем пересекая текущее разбиение и эелементы $A$).
\end{enumerate}
Заметим, что любой элемент из $A$ -- конечное объединение каких-то элементов рашего разбиения. Пусть $Q = \lbrace \Omega_1, \Omega_2, \ldots, \Omega_n \rbrace$, тогда $A \subseteq \mathcal{P}(Q)$.

С другой стороны $\mathcal{P}(Q) \subseteq A$, так как любой элемент $\mathcal{P}(Q)$ -- конечное объедиенение каких-то элементов $Q$, а значит элемент должен лежать и в $A$.

Тогда мы имеем, что $A = \mathcal{P}(Q)$. Откуда следует, что $A$ пораждается конечным разбиением $\Omega$ и $|A| = 2^{|Q|}$.
\end{solution}
%\begin{task}{9}
Есть поток сигма-алгебр $F_1 \subset F_2 \subset \ldots$ Является ли $\sigma$-алгеброй объединение всех этих систем?
\end{task}
\begin{solution}
    Пусть $F_n$ -- $\sigma$-алгебра, порожденная $\{\varnothing, \{1\}, \{2\}, \ldots, \{n\}, \{n + 1, n + 2, \ldots\}\}$.
    
    Пусть $X$ -- множество всех нечентых чисел из $\mathbb{N}$.
    
    Пусть $F = \underset{n=1}{\overset{\infty}{\bigcup}} F_n$.
    
    Тогда $F$ -- не $\sigma$-алгебра, так как $\forall x \in X$ выполено,что $x \in F$, но $X \notin F$, а значит не выполняется замкнутость относительно счетного объединения ($X \sim \mathbb{N}$).
\end{solution}
%\begin{task}{10}
Существует ли такая счетная система подмножеств $R$, что $\sigma (R)$ -- борелевская $\sigma$-алгебра?
\end{task}

\begin{solution}
Для начала рассмторим $\mathfrak{B}\left(\mathbb{R}\right)$. Заметим, что $\forall X \in \mathfrak{B}\left(\mathbb{R}\right)$ выполено: \[ X = \underset{n}{\bigsqcup} \langle a_n, b_n \rangle,\] где $a_n, b_n \in \mathbb{R}$ и $a_n \leq b_n$.

Пусть $R = \lbrace (-\infty, a) : ~ a \in \mathbb{Q} \rbrace \sqcup \lbrace (-\infty, a] : ~ a \in \mathbb{Q} \rbrace \sqcup \lbrace (a, +\infty ) : ~ a \in \mathbb{Q} \rbrace \sqcup \lbrace [a, +\infty ) : ~ a \in \mathbb{Q} \rbrace$

Докажем, что $\sigma (R) = \mathfrak{B}(\mathbb{R})$.
Доказательство:
\begin{enumerate}
    \item $\sigma (R)$ содержит все промежутки вида $\langle a, b \rangle$, где $a, b \in \mathbb{Q}$.
    
    \item из $1.$ следует, что $\sigma (R)$ содержит все множества вида $\underset{n}{\bigsqcup} \langle a_n, b_n \rangle$, где $a_n, b_n \in \mathbb{Q}$ и $a_n \leq b_n$.
    
    \item $\sigma (R)$ содержит все промежутки вида $\langle a, b \rangle$, где $a, b \in \mathbb{R}$, так как рассмторим последовательности ${a_n}$, ${b_n}$ из $\mathbb{Q}$ такие, что ${a_n}$ возрастает, ${b_n}$ убывает и $\underset{n\to\infty}{\lim} a_n = a$ и $\underset{n\to\infty}{\lim} b_n = b$. Тогда $\langle a, b \rangle = \underset{n}{\bigcap} \langle a_n, b_n \rangle$, откуда следует, что $\langle a, b \rangle$ лежит в $\sigma (R)$
    
    \item из того, что $\langle a, b \rangle$ лежит в $\sigma (R)$ следует, что $\sigma (R)$ содержит все множества вида $\underset{n}{\bigsqcup} \langle a_n, b_n \rangle$, где $a_n, b_n \in \mathbb{Q}$ и $a_n \leq b_n$. Здесь все границы интервалов действительны.
\end{enumerate}

Тогда мы имеем, что $\mathfrak{B}(\mathbb{R}) \subseteq \sigma(R)$.

Из построения $R$ следует, что $\sigma (R) \subseteq \mathfrak{B}(\mathbb{R})$.

Тогда имеем, что $\mathfrak{B}(\mathbb{R}) = \sigma(R)$. А по построению $R \sim \mathbb{Q} \sim \mathbb{N}$.
\end{solution}

%\section{Мера.}
% \begin{task}{1}
Построить пример полукольца $S$ и такой функции $\varphi : S \to [0; +\infty)$, что $\forall A, B \in S: A \cap B  = \varnothing \text{ и } C = A \sqcup B \in S$
выполнено равенство  $\varphi(C) = \varphi(A) + \varphi(B)$, но $\varphi$ --- не мера на $S$.
\end{task}
\begin{solution}
Рассмотрим систему множеств $S = \{\varnothing, \{1\}, \{2\}, \{3\}, \{4\}, \{1, 2\}, \{1, 2, 3, 4\} \}$. Можно показать, что данная система является полукольцом. Определим на $S$ функцию $\varphi$ следующим образом:

$\varphi(\varnothing) = 0$

$\varphi(\{1\}) = \varphi(\{2\}) = \varphi(\{3\}) = \varphi(\{4\}) = 1$

$\varphi(\{1, 2\}) = 2$

$\varphi(\{1, 2, 3, 4\}) = 3$

Тогда для данной функции выполняется равенство $\varphi(\{1, 2\}) = \varphi(\{1\}) + \varphi(\{2\})$, однако $\varphi$ не является мерой на $S$, т.к. $\varphi(\{1, 2, 3, 4\}) = 3 \neq 4 = \varphi(\{1, 2\}) + \varphi(\{3\}) + \varphi(\{4\})$.
\end{solution}
%\begin{task}{2}
Пусть $m$ --- мера на полукольце $S$. Докажите, что
\begin{enumerate}[(a)]
    \item если множества $A$ и $B$ принадлежат $S$ и $B \subseteq A$, то $m(B) \leqslant m(A)$.
    \item $m(\varnothing) = 0$
    \item Если $A, B, A \cup B \in S$, то $m(A \cup B) = m(A) + m(B) - m (A\cap B)$
    \item Если $A, B, A \triangle B \in S$ и $m(A \triangle B) = 0$, то $m(A) = m(B)$
\end{enumerate}
\end{task}
\begin{solution}
\begin{enumerate}[(a)]
    \item Из определения полукольца следует, что $\exists B_1, \dots, B_n \in S :$ 
    \begin{equation*}
        A \setminus B  = \bigsqcup_{i = 1}^{n} B_i
    \end{equation*}
    Тогда из аддитивности меры получим, что
    \begin{equation}\label{measureEq}
        m(A) = m(B) + \sum_{i = 1}^{n} m(B_i)
    \end{equation}
    В силу неотрицательности меры~\eqref{measureEq} влечет за собой неравентво $m(B) \leqslant m(A)$.
    \item Т.к $\varnothing \cap \varnothing = \varnothing$, то $\varnothing = \varnothing \sqcup \varnothing \rightarrow m(\varnothing) = m(\varnothing \sqcup \varnothing) = m(\varnothing) + m(\varnothing) \rightarrow m(\varnothing) = 0$.
    \item Т.к $A \cup B = (A \setminus B) \sqcup B$ и $A \cup B \in S$, то $\exists B_1, \dots, B_n \in S :$
    \begin{equation*}
        A \setminus B  = \bigsqcup_{i = 1}^{n} B_i
    \end{equation*}
    Тогда в силу аддитивности меры
    \begin{equation}\label{2cfirstEq}
        m(A \cup B) = m(B) + \sum_{i = 1}^{n} m(B_i)
    \end{equation}
    С другой стороны, $A = (A \cap B) \sqcup (A \setminus B)$ и $A \cup B \in S$ по определению полукольца, поэтому
    \begin{equation}\label{2csecondEq}
        m(A) = m(A \cap B) + \sum_{i = 1}^{n} m(B_i)
    \end{equation}
    Вычитая~\eqref{2cfirstEq} из~\eqref{2csecondEq}, получим
    \begin{equation*}
        m(A \cup B) - m(A) = m(B) - m(A \cap B) \rightarrow m(A \cup B) = m(A) + m(B) - m(A\cap B),
    \end{equation*}
    ч.т.д.
    \item Пусть $A \setminus B = \bigsqcup_{i = 1}^{n} A_i,\hspace{3mm} B \setminus A = \bigsqcup_{j = 1}^{k} B_j; \hspace{5mm}A_j, B_j \in S$ (аналогично задачам 1 и 3). Тогда, т.к $A \triangle B  = (A \setminus B) \sqcup (B \setminus A)$
    \begin{equation*}
        m(A \triangle B) = \sum_{i = 1}^{n} m(A_i) + \sum_{j = 1}^{k} m(B_j)
    \end{equation*}
    Т.к $m(A \triangle B) = 0$ и $m$ неотрицательна,
    \begin{equation}\label{equaltoZero}
    \sum_{i = 1}^{n} m(A_i) = \sum_{j = 1}^{k} m(B_j) = 0.
    \end{equation}
    Воспользовавшись тем, что $A = (A \cap B) \sqcup (A \setminus B)$ и $B = (A \cap B) \sqcup (B \setminus A)$ и равенством~\eqref{equaltoZero}, получим, что $m(A) = m(A\cap B) = m(B)$.
\end{enumerate}
\end{solution}
%\begin{task}{3}
\begin{enumerate}[(a)]
\item Пусть $\mathcal{S}$ --- полукольцо с мерой $m$, а $\mathcal{S}_1 = \{A \in \mathcal{S}: m(A) = 0\}$. Доказать, что  $\mathcal{S}_1$ --- полукольцо.
\item Пусть $\mathcal{R}$ --- кольцо с мерой $m$, а $\mathcal{R}_1 = \{A \in \mathcal{R}: m(A) = 0\}$. Доказать, что  $\mathcal{R}_1$ --- кольцо.
\item Пусть $\mathcal{A}$ --- алгебра с мерой $m$, а $\mathcal{A}_1 = \{A \in \mathcal{A}: m(A) = 0\}$. Верно ли, что  $\mathcal{A}_1$ --- алгебра?
\end{enumerate}
\end{task}
\begin{solution}
\begin{enumerate}[(a)]
\item\label{firstPt}
Проверим выполнение соответствующих определений:
\begin{enumerate}[1)]
    \item \varnothing \in $\mathcal{S}_1$, т.к $m(\varnothing) = 0$.
    \item Пусть $A, B \in \mathcal{S}_1$. Тогда $A, B \in \mathcal{S}$, $m(A) = 0$ и $m(A) = 0$.
    Из этого следует, что $m(A \cap B) = 0$, т.е $A \cap B \in \mathcal{S}_1$.
    \item Пусть $A, B \in \mathcal{S}_1, A \subset B$. Тогда $\exists B_1, \dots, B_n \in \mathcal{S}: A = B \sqcup B_1 \sqcup \dots \sqcup B_n$. Т.к. $A \in \mathcal{S}_1$, то $m (A) = 0$. Но тогда, т.к $B, B_1, \dots, B_n \subset A$, меры всех множеств $B, B_1, \dots, B_n$ равны нулю. Следовательно $B, B_1, \dots, B_n \in \mathcal{S}_1$.  
\end{enumerate}
\item Аналогично предыдущему пункту:
\begin{enumerate}[1)]
    \item \varnothing \in $\mathcal{R}_1$ (т.к $m(\varnothing) = 0)$, следовательно, $\mathcal{R}_1$ непусто.
    \item Проверка пересечения множеств аналогична пункту~\eqref{firstPt}.
\end{enumerate}
\item Пусть $\mathcal{A}$ — все подмножества отрезка $[0, 1]$, измеримые по Жордану, тогда в $\mathcal{A}_1$ будут все точки отрезка $[0, 1]$, но в $\mathcal{A}_i$ не будет множества, содержащего их все, поэтому в $\mathcal{A}_1$ не будет единицы.
\end{enumerate}
\end{solution}

%\begin{task}{4}
Пусть $m$ -- $\sigma$-аддитиваня мера на полукольце $S$, множества $A, A_1, \ldots, A_i, \ldots$ придлежат $S$, причем $A_1 \supseteq A_2 \supseteq \ldots$ и
\[
A = \bigcap_{i = 1}^{\infty}A_i.
\]
Доказать, что
\[
m(A) = \lim\limits_{i \to \infty}m(A_i).
\]
Это свойство меры называется \textit{непрерывностью.}
\end{task}
\begin{solution}
Рассмотрим $\lbrace A_n \rbrace \in S$ такую,  что $A_1 \supseteq A_2 \supseteq A_3 \ldots$, и $A = \underset{n=1}{\overset{\infty}{\bigcap}} A_n$.

Обозначим $B_i = A_i \setminus A_{i+1}$, тогда $A_1 \setminus A = 
\bigsqcup\limits_i B_i = \bigsqcup\limits_i \bigsqcup\limits_j C_{i, j}$, где 
$C_{i, j} \in S$

\begin{gather*}
m(A_1) - m(A) = \sum_i \sum_j m(C_{i, j}) = 
\lim_{k \to \infty} \sum_{i=1}^{k-1} \sum_{j = 1}^{j_i} m(C_{i,j}) = \lim_{k 
\to \infty} \sum\limits_{i=1}^{k-1} \left( m(A_i) - m(A_{i+1}) \right) =\\ = 
m(A_1) - \underset{i \to \infty}{\lim} m(A_i)
\end{gather*}
чтд.
\end{solution}
%\begin{task}{5}
Пусть $m$ -- мера на кольце $R$ и для любых таких множеств $A, A_1, \ldots, A_i, \ldots$ из $R$, что $A_1 \supseteq A_2 \supseteq \ldots$ и
\[
A = \bigcap_{i = 1}^{\infty}A_i.
\]
выполнено равенство
\[
m(A) = \lim\limits_{i \to \infty}m(A_i).
\]
Доказать, что $m$ -- $\sigma$-аддитивная мера. Или иначе: доказать $\sigma$-аддитивность непрерывной меры.\newline
Показать, что это утверждение может не быть справедливым для меры на полукольце. 
\end{task}
\begin{solution}

\end{solution}
%\begin{task}{6}
Построить пример меры на полукольце, которая не является $\sigma$-аддитивной.
\end{task}

\begin{solution}
Пусть $S = \lbrace (a, b] \cap \mathbb{Q} : ~ a, b \in \mathbb{R},~a \leq b  \rbrace$.
Пусть $\mu((a, b) \cap \mathbb{Q}) = b - a$.

Понятно, что $S$ -- полукольцо, а $\mu$ -- мера на нём.

Докажем, что $\mu$ -- не $\sigma$-аддитивна:

Рассмторим $E = (0, 1] \cap \mathbb{Q}$. Так как $E \sim \mathbb{N}$, то занумеруем точки $E$ : $E = \lbrace r_n \rbrace$.

Построим последовательность множеств $\lbrace A_n \rbrace$ по индукции:
\begin{enumerate}
    \item
    $A_1 = (max(0, r_1 - \frac{1}{2^3}), r_1] \cap \mathbb{Q}$.
    
    \item
    Пусть $k_{n+1} = inf \lbrace j \geq 1 : ~ r_j \notin \underset{k=1}{\overset{n}{\bigsqcup}} A_k \rbrace$
    
    \item
    Положим $A_{n+1} = (r_{n_{k+1}} - x_{n+1}, r_{n_{k+1}}] \cap \mathbb{Q}$, где $x_{n+1} \in (0, 2^{-n-3})$ такое, что $A_{n+1} \bigcap \left( \underset{k=1}{\overset{n}{\bigsqcup}} A_k \right) = \varnothing$.
\end{enumerate}

Тогда $E = \underset{n=1}{\overset{\infty}{\bigsqcup}} A_n$.

Найдем меру $E$:
\begin{enumerate}
    \item $\mu(E) = 1 - 0 = 1$
    
    \item $\mu(E) = \mu(\underset{n=1}{\overset{\infty}{\bigsqcup}} A_n) = \sum\limits_{n=1}^{\infty} \mu(A_n) = \sum\limits_{n=1}^{\infty} x_n \leq \sum\limits_{n=1}^{\infty} \frac{1}{2^{2+n}} = \frac{1}{4}$
\end{enumerate}

Таким образом, мы получили, что $\mu$ -- не $\sigma$-аддитивная мера на $S$.
\end{solution}
%\begin{task}{7}
Пусть $m$ --- $\sigma$-аддитивная мера на полукольце $S$, множества 
$A, A_1, \dots A_i, \dots$ принадлежат $S$, причем $A_1 \subseteq A_2 \subseteq \dots$ и
$$A = \bigcup_{i=1}^{\infty} A_i.$$
Доказать, что 
$$m(A) = \lim_{i \to \infty} m(A_i).$$
\end{task}

\begin{solution}
Определим для каждого $i \geqslant 1$ множество $B_i = A_{i+1} \setminus A_i$. Тогда 
$$A \setminus A_1 = \bigsqcup_{i=1}^{\infty} B_i.$$
Так как $S$ --- полукольцо, то $\forall k \text{ } \exists C_1, \dots, C_{j_k} \in S$, такое что
$$B_k = \bigsqcup_{j=1}^{j_k}C_j.$$
Отсюда получим, что
$$ m(A) - m(A_1) = \sum_{i=1}^{\infty} \sum_{j=1}^{j_i} m(C_j) = \lim_{i \to \infty} \sum_{k = 1}^{i - 1} \sum_{j=1}^{j_k} m(C_j) = \lim_{i \to \infty} (\sum_{k=1}^{i-1}(m(A_{k+1}) - m(A_k))) = $$
$$ = \lim_{i \to \infty} (m(A_{i}) - m(A_1)) = \lim_{i \to \infty} m(A_i) - m(A_1).$$
Из этого следует, что 
$$m(A) = \lim_{i \to \infty} m(A_i).$$
\end{solution}
%\begin{task}{8}
Пусть $m$ --- мера на кольце $R$ и для любых таких множеств 
$A, A_1, \dots A_i, \dots$ из $R$, что $A_1 \subseteq A_2 \subseteq \dots$ и
$$A = \bigcap_{i=1}^{\infty} A_i$$
выполняется 
\begin{equation}\label{limEq}
    m(A) = \lim_{i \to \infty} m(A_i).
\end{equation}
Докажите, что $m$ --- $\sigma$-аддитивная мера. Показать, что это утверждение может не быть справедливым для меры на полукольце.
\end{task}

\begin{solution}
Пусть $B, B_1, \dots B_i, \dots$ принадлежат $R$ и 
$$B = \bigsqcup_{i=1}^{\infty} B_i.$$ Определим множества
$$ C_l = \bigsqcup_{i=1}^{l} B_i = B \setminus \left(\bigsqcup_{i = l+1}^{\infty} B_i \right), l = 1, 2, \dots.$$
Тогда $C_1 \subseteq C_2 \subseteq \dots$ и
$$ \bigcup_{l=1}^{\infty} C_l = B.$$
Тогда множества $C_1, C_2, \dots$ удовлетворяют условию~\eqref{limEq}, следовательно
$$m(B) = \lim_{l \to \infty} m(C_l).$$
Если нашлось такое $n$, что $m(B_n) = +\infty$, то очевидным образом 
$$m(B) = +\infty = \sum_{i=1}^{\infty} m(B_i).$$
Иначе
$$ m(B) = \lim_{l \to \infty} m(C_l) = \lim_{l \to \infty} m\left( \bigsqcup_{i=1}^{l} B_i \right) = \lim_{l \to \infty} \sum_{i=1}^{l} m(B_i) = \sum_{i=1}^{\infty} m(B_i),$$ что и требовалось доказать.

Рассмотрим полукольцо $S = \{\langle a, b \rangle \cap \mathbb{Q}: 0 \leqslant a \leqslant b \leqslant 1\} \cup \{\varnothing\}$ и меру $m(\langle a, b \rangle) = b - a$ (аналогично задаче 5). Покажем, что $m$ непрерывна, но не $\sigma$-аддитивна.

Пусть $\mathbb{Q} \cap [0, 1] = \{r_n\}_{n=1}^{\infty}$. Заметим, что для любого $n \text{ } m(\langle r_n, r_n \rangle) = 0$. Поэтому
$$ 1 = m(\mathbb{Q} \cap [0, 1]) = m\left(\bigsqcup_{i=1}^{\infty} m(\langle r_n, r_n \rangle) \right) \neq \sum_{i=1}^{\infty} m(\langle r_n, r_n \rangle) = 0,$$
т.е $m$ не $\sigma$-аддитивна.

Пусть теперь $A, A_1, \dots \in S: A_1 \subseteq A_2 \subseteq \dots$ и
$$ A = \bigcup_{i=1}^{\infty} A_i.$$
Если $A = \langle a, b \rangle, A_i = \langle a_i, b_i \rangle$, то $a_i \to a, b_i \to b$. Следовательно,
$$\lim_{i \to \infty} m(A_i) = \lim_{i \to \infty} (b_i - a_i) = b - a = m(A),$$
т.е мера $m$ непрерывна.
\end{solution}
%\begin{task}{9}
Показать, что в случае $\sigma$-конечной меры понятия непрерывности и $\sigma$-аддитивности не равносильны.
\end{task}

\begin{solution}
Пусть $m$ --- классическая мера Лебега на $\mathbb{R}$. Пусть $A_i = [i, +\infty)$. Тогда
$$ \bigcap_{i=1}^{\infty} = \varnothing .$$
Однако
$$ \lim_{i \to \infty} m(A_i) = +\infty \neq 0 = m(\varnothing). $$
\end{solution}

%\section{Внешняя мера. Мера Лебега.}
%\begin{task}{1}
Доказать, что если вопреки определению верхней меры, мера $m$ не $\sigma$-аддитивна, то найдется такое множество $A \in S$, для которого $\mu^{*}(A) < m(A)$. 
\end{task}
\begin{solution}
По условию, $\exists A \in S \text{ и } \{ A_n \}_{n=1}^{\infty} \in S$ такие, что 

\begin{equation}
    A = \bigsqcup_{n=1}^{\infty}{A_n} \text{ и } m(A) \not = \sum_{n=1}^{\infty}{m(A_n)}.
\end{equation}

По свойству $m$, получаем:

\begin{equation}
    \mu^{*} (A) \leq \sum_{n=1}^{\infty}{m(A_n)} < m(A).
\end{equation}
\end{solution}

%\begin{task}{2}
Пусть $A \in X$ и $B \in X$. Показать, что $\mu^{*}(A \cup B) \leq \mu^{*}(A) + \mu^{*}(B)$.
\end{task}
\begin{solution}
Пусть дано $\varepsilon > 0$. Тогда для $A$ и $B$ можно найти такие $\{ A_n \}_{n=1}^{\infty} , \{ B_n \}_{n=1}^{\infty} \in~S$, что:

\begin{align*}
    &A \subseteq \bigcup_{n=1}^{\infty}{A_n}, \mu^{*}(A) \geq \sum_{n=1}^{\infty}{m(A_n)} - \dfrac{\varepsilon}{2}; \\
    &B \subseteq \bigcup_{n=1}^{\infty}{B_n}, \mu^{*}(B) \geq \sum_{n=1}^{\infty}{m(B_n)} - \dfrac{\varepsilon}{2}.
\end{align*}

Тогда

\begin{equation}
    A \cup B \subseteq \bigcup_{n=1}^{\infty}{\left( A_n \cup B_n \right)}.
\end{equation}

откуда следует:

\begin{equation}
    \mu^{*}\left( A \cup B \right) \leq \sum_{n=1}^{\infty}{m(A_n)} + \sum_{n=1}^{\infty}{m(B_n)} \leq \mu^{*}(A) + \mu^{*}(B) + \varepsilon.
\end{equation}

В силу произвольности $\varepsilon$ получаем требуемое неравенстов:

\begin{equation}
    \mu^{*}(A \cup B) \leq \mu^{*}(A) + \mu^{*}(B).
\end{equation}
\end{solution}
%\begin{task}{3}
Пусть $A \subseteq X$ и $B \subseteq X$. Докажите, что
$$ \mu^*(A \cup B) + \mu^* (A \cap B) \leqslant \mu^*(A) + \mu^*(B).$$
\end{task}

\begin{solution}

\end{solution}
%\begin{task}{4}
Докажите, что если $E \subseteq \mathbb{R}$ измеримо, то для любого $A \subseteq \mathbb{R}$ выполено \[ \mu^*(A) = \mu^*(A \setminus E) + \mu^*(A \cap E) \]
\end{task}

\begin{solution}
\begin{enumerate}
    \item
    Пусть $A \cap E \subseteq \bigcup B_i$, $A \setminus E \subseteq \bigcup C_j$, тогда $A \subseteq \left( \bigcup B_i \right) \cup \left( \bigcup C_j \right)$.
    
    Тогда: $\mu^*(A) \leq \sum m(B_i) + \sum m(C_j) \rightarrow \mu^*(A) \leq \mu^*(A \cup E) + \mu^*(A \setminus E)$ (По предельному переходу).
    
    \item
    Пусть $E \subseteq \bigcup E_i^\delta$, причем $\mu^*(E \Delta \bigcup E_i^\delta) < \delta$; $A \subseteq \bigcup A_j$.
    
    Тогда: $A \setminus \bigcup E_i^\delta \subseteq \left(\bigcup A_j\right) \setminus \left(\bigcup E_i^\delta\right)$
    
    и $\mu^*(A \setminus \bigcup E_i^\delta) \leq m((\bigcup A_j) \setminus (\bigcup E_i^\delta)) = m(\bigcup A_j) - m((\bigcup A_j) \cap (E_i^\delta))$.
    
    $\mu^*(A \cap (\bigcup E_i^\delta)) \leq m((\bigcup A_j) \cap (\bigcup E_i^\delta))$.
    
    Тогда: $\mu^*(A \cap \bigcup E_i^\delta) + \mu^*(A \setminus \bigcup E_i^\delta) \leq m(\bigcup A_i) \rightarrow \mu^*(A \cap \bigcup E_i^\delta) + \mu^*(A \setminus \bigcup E_i^\delta) \leq \mu^*(A)$ тогда используя предельный переход по $\delta$ имеем: $\mu^*(A \cap E) + \mu^*(A \setminus E) \leq \mu^*(A)$
\end{enumerate}

Таким обоазом, $\mu^*(A) = \mu^*(A \cap E) + \mu^*(A \setminus E)$.
\end{solution}
%\begin{task}{5}
  Доказать, что в случае классической меры Жордана система множеств $\mathfrak{M}_J$ не
  является $\sigma$-алгеброй. Привести пример меры $m$, когда она является $\sigma$-алгеброй.
\end{task}

\begin{solution}
  Пусть $\mathbb{Q}_{[0,1]} = \{ r_n \}_{n=1}^{\infty}$ --- множество рациональных чисел
  отрезка $[0,1]$. Тогда $\forall n: \{ r_n\} \in \mathfrak{M}_J, \mu_J (\{ r_n \}) = 0$, но 
  $\mathbb{Q}_{[0,1]} = \bigsqcup\limits_{n = 1}^{\infty} \{ r_n \} \not\in \mathfrak{M}_J$ (не измеримо по Жордану).
  
  Если вместо классической меры взять в качестве $m$ взять тождественный ноль на $\sigma$-алгебре $2^\mathbb{R}$, то $\mu_J^*$ на
  всех подмножествах $\mathbb{R}$ будет равняться $0$, откуда будет следовать, что $\mathfrak{M}_J = 2^{\mathbb{R}}$. (???)
\end{solution}
%\begin{task}{6}
Пусть множество $E \subseteq \mathbb{R}$ имеет положительную меру Лебега. Докажите, что множество $E - E = \{x - y: x, y \in E\}$ содержит интервал с центром в 0.
\end{task}
\begin{solution}
Пусть дано $\varepsilon > 0$. Рассмотрим такие множества $K$ - компакт, $O$ - октрытое множество, что:
\begin{equation}
    K \subseteq E \subseteq O, \mu(K) + \varepsilon > \mu(E) > \mu(O) - \varepsilon.
\end{equation}
Пусть также $2 \mu(K) > \mu(O)$.

Возьмем такую $\delta$, что $\forall k \in K \Rightarrow U_{\delta}(k) \subseteq O$. Тогда $U_{\delta / 2}(k) \subseteq U_{\delta}(k) \subseteq O$. Рассмотрим следующее множество $\{U_{\delta / 2}(k): k \in K\}$ -- открытое покрытие компакта. Тогда в нем можно выбрать конечное покрытие $\{U_{\delta / 2}(k_1), \dots, U_{\delta / 2}(k_n)\}$.

Тогда ('+' - сумма Минковского):
\begin{equation}
    K + (-\delta / 2; \delta / 2) \subset \bigcup_{i=1}^{n}{U_{\delta / 2}(k_i)} + (-\delta / 2; \delta / 2) \subset \bigcup_{i=1}^{n}{U_{\delta}(k_i)} \subset O.
\end{equation}

Покажем, что $\forall v \in (-\delta / 2; \delta / 2) \Rightarrow (K + v) \cap K \not = \emptyset$. Пусть это не так:
\begin{align*}
& \forall v \in (-\delta / 2; \delta / 2) \Rightarrow \left(K + v\right) \cup K \subset K + (-\delta / 2; \delta / 2) \subset O \\
    & \mu(K + v) + \mu(K) < \mu(O) \\
    &\exists v \in (-\delta / 2; \delta / 2): (K + v) \cap K = \emptyset \Rightarrow A = (K + v) \sqcup K\\
    &\mu(A) = \mu(K + v) + \mu(K) = 2 \mu(K) < \mu(O) . 
\end{align*}
Что противоречит выбору $K$ и $O$. 

В итоге получаем:
\begin{equation}
    \forall v \in (-\delta / 2; \delta / 2) \Rightarrow \exists k_1, k_2 \in K \subseteq E: v + k1 = k2. \implies (-\delta / 2; \delta / 2) \subset E - E.
\end{equation}
\end{solution}

%\begin{task}{7}
Построить такие неизмеримые относительно классической меры Лебега на $[0;1]$ множества $A_1, A_2$, что $A_1 \cup A_2$ измеримо.
\end{task}
\begin{solution}
Возьмем в качестве $A_1$ -- множество Витали на $[0;1]$, $A_2$ - его дополнение. Множество Витали не измеримо. Пусть его дополнение измеримо. Но тогда множество Витали было бы измеримым. Следовательно, $A_2$ неизмеримо. $A_1 \cup A_2 = E = [0;1]$, то есть их объединение измеримо.

\begin{definition}
Множество Витали - неизмеримое по Лебегу множество. Его построение:

Рассмотрим такое отношение эквивалентности $\sim$: $x \sim y$, если $x - y \in \mathbb{Q}$. Это отношение разбивает $[0;1]$ на классы эквивалентности. Выберем по представителю в каждом классе. Полученное множество $A$ будет неизмеримым.
\end{definition}

\begin{proposition}
Множество Витали неизмеримо.
\end{proposition}
\begin{proof}
Занумеруем все рациональные числа на $[-1;1]$. Получим следующую последовательность $\{r_n\}_{n=1}^{\infty}$. Пусть $A_n = A + r_n$. Докажем, что полученные множества не пересекаются. Пусть это не так. Тогда 
\begin{equation}
    \exists x = a_n + r_n = a_m + r_m, n \not = m.
\end{equation}
Но тогда $a_m - a_n \in \mathbb{Q}$, а значит $a_n$ и $a_m$ лежат в одном множестве. Противоречие.

Пусть в $A_n$ находится измеримое множество $C_n$ с мерой $d > 0$. Тогда
\begin{equation}
    \forall m \exists C_m \subseteq A_m: \mu(C_m) = d.
\end{equation}
Но 

\begin{equation}
    \bigsqcup_{n=1}^{\infty}{C_n} \subseteq \bigsqcup_{n=1}^{\infty}{A_n} \subseteq [-1; 2],
\end{equation}

откуда

\begin{equation}
    \sum_{n=1}^{\infty}{\mu(C_n)} = \sum_{n=1}^{\infty}{d} \leq 3.
\end{equation}

Противоречие.

С другой стороны,

\begin{equation}
    [0;1] \subseteq \bigsqcup_{n=1}^{\infty}{A_n}.
\end{equation}

Если мера $A_n$ равна нулю, то и сумма тоже будет равна нулю. Но $\mu([0;1]) = 1$. Противоречие.

Следовательно, множество Витали неизмеримо.

\end{proof}

\end{solution}

%\begin{task}{8}
Пусть $A \in \mathfrak{M}, \mu(A) = 0$ и $B \subset A$. Докажите, что
$B \in \mathfrak{M}$ и $\mu(B) = 0$.
\end{task}

\begin{solution}
Так как $B \subset A$, то $\mu^*(B) \leqslant \mu^*(A) = \mu(A) = 0$. Докажем, что если $\mu^*(B)= 0$, то $B \in \mathfrak{M}$, т.е $\forall \varepsilon > 0 \exists A_{\varepsilon}: \mu^*(A \triangle A_{\varepsilon}) < \varepsilon$. Взяв в определении $A_{\varepsilon} = \varnothing$ для всех $\varepsilon > 0$, получим, что $B \in \mathfrak{M}$ и $\mu(B) = \mu^*(B) = 0$
\end{solution}
%\begin{task}{9}
Пусть $\mu$ --- классическая мера Лебега на $[0, 1]$. Построить такую последовательность $\{A_i\}$ множеств из $\mathfrak{M}$, что
$$ \mu(\liminf_{i \to \infty} A_i) < \lowlim_{i\to\infty} \mu(A_i) $$
\end{task}

\begin{solution}
Пусть $A_{2n-1} = \left(0, \dfrac{1}{2}\right), A_{2n} = \left(\dfrac{1}{2}, 1\right), n \in \mathbb{N}$. Тогда $\mu(A_n) = \dfrac{1}{2}$ при каждом $n$. Но т.к $A_{2n-1} \cap A_{2n} = \varnothing$, 
$$ \mu(\liminf_{i \to \infty} A_i) = \mu(\varnothing) = 0.$$
\end{solution}

\section{Измеримые функции.}
\begin{task}{1}
Доказать (не опираясь на критерий измеримости), что если функции $f(x)$ и $g(x)$ измеримы, то и множество $\lbrace x : f(x) < g(x) \rbrace$ измеримо. Получить отсюда, что $f(x) + g(x)$ - измеримая функция.
\end{task}
\begin{solution}
$\lbrace x : f(x) < g(x) \rbrace = \bigcup_{n=1}^\infty(\lbrace x : f(x) < r_n \rbrace \cap \lbrace x: r_n < g(x)\rbrace)$, где $r_n$ - последовательность рационых чисел. Так как справа все множества измеримы, то и исходное тоже измеримо.

$(f + g)(x) : \{x \in X \mid f(x) + g(x) > C\} = \{x \in X \mid f(x) > C - g(x)\}$, где $f(x)$ и $g(x)$ -- измеримы, значит $f + g$ измеримая.
\end{solution}
\begin{task}{2}
Пусть $(X, M, \mu)$ -- измеримое пространство, $A \subseteq X$ и $f(x) = \mathbb{I}_A(x)$. Доказать, что $f(x)$ измерима на $X$ тогда и только тогда, когда $A \in M$.
\end{task}
\begin{solution}
%Пусть $f$ -- измерима, тогда $\forall c E_c = {x \in X : f(x) > c}$ -- измеримо. Тогда при $c = 0 E_c = A$, откуда следует, что $A$ - измеримо.

%Обратно, пусть $A$ -- измеримо, тогда $E_c$ может быть равно $X$, $A$, $\varnothing$, которые измеримы, тогда измерима и $f(x)$.
$\forall c < 0 \Rightarrow f^{-1}((c, +\infty)) = X \in M$.

$\forall c > 1 \Rightarrow f^{-1}((c; +\infty)) = \emptyset \in M$

$\forall c \in [0, 1] \Rightarrow f^{-1}((c;+\infty)) = A. $ Следовательно, функция измерима тогда и только тогда, когда $A \in M$.
\end{solution}
\begin{task}{3}
Пусть $\mu$ -- классическая мера Лебега на $[0; 1]$. Построить такую неизмеримую функцию $f : [0; 1] \rightarrow \mathbb{R}$, что для любого $c \in \mathbb{R}$ множество $f^{-1}(\lbrace c \rbrace)$ измеримо. 
\end{task}
\begin{solution}
Пусть $E$ - неизмеримое множество. Тогда:
\begin{equation}
f(x) = \begin{cases}
	x, \text{если $ x \in E$} \\
    -x, \text{если $x \not \in E$}
\end{cases}.
\end{equation}
$f(x)$ - неизмеримо. Действительно, $f^{-1}((0, +\infty))$ не измеримо. Но $\forall c \in \mathbb{R} \Rightarrow f^{-1}(\{c\}) - \text{точка} \Rightarrow f^{-1}(\{c\}) - \text{измеримо}$.
\end{solution}
\begin{task}{4}
Пусть $(X, M, \mu)$ --- измеримое пространство, $\{a_n\}_{n=1}^{\infty}$ --- всюду плотное множество в $\mathbb{R}$, а функция $f: X \rightarrow \mathbb{R}$ такова, что
$$ \forall n f^{-1}((a_n, +\infty)) \in M.$$
Доказать, что $f(x)$ измерима на $X$.
\end{task}

\begin{solution}
Возьмем произвольное $c \in \mathbb{R}$. Т.к $\{a_n\}$ всюду плотно в $\mathbb{R}$, то найдется подпоследовательность $\{a_{n_k}\}_{k=1}^{\infty}$, такая что $a_{n_k} \downarrow c$ при $k \to \infty$. Тогда
$$ f^{-1}((c, +\infty)) = \bigcup_{k=1}^{\infty} f^{-1}((a_{n_k}, +\infty)) \in M.$$
\end{solution}
\begin{task}{5}
Построить функцию $f(x)$ на $[0, 1]$, измеримую на $[0, 1]$ относительно классической меры Лебега на $[0, 1]$, но разрывную в каждой точке.
\end{task}

\begin{solution}
Функция Дирихле на $[0, 1]$
$$ f(x) =
\begin{cases}
1, \text{если $x \in [0, 1] \cap \mathbb{Q}$} \\
0, \text{если $x \in [0, 1] \setminus \mathbb{Q}$}
\end{cases} $$
измерима, т. к $f^{-1}(\{1\}) = [0, 1] \cap \mathbb{Q}, f^{-1}(\{0\}) = [0, 1] \setminus \mathbb{Q}$ --- измеримые множества. При этом $f$ разрывна в каждой точке.
\end{solution}
\begin{task}{6}
Пусть $(X, M, \mu)$ --- полное измеримое пространство (т.е мера $\mu$ полна), а $f(x)$ --- измеримая функция на $A$. Пусть $g(x)$ --- функция, эквивалентная $f(x)$. Доказать, что $g(x)$ --- измеримая на $A$ функция.
\end{task}

\begin{solution}
Пусть $A_0 = \{x \in A: f(x) \neq g(x) \}$. Т.к мера $\mu$ полна, то любое множество $B \subseteq A_0$ измеримо. Для каждого $c \in \mathbb{R}$ определим множества $B_1 = \{x: f(x) \leqslant c, g(x) > c$ и $B_2 = \{x: g(x) \leqslant c, f(x) > c \}$. Множества $B_1, B_2 \subseteq A_0$ и для каждого $c \in \mathbb{R}$ выполнено равенство $$g^{-1}((c, +\infty]) = (f^{-1}((c, +\infty]) \cup B_1) \setminus B_2.$$ Отсюда $g(x)$ измерима на $A$.
\end{solution}
\begin{task}{7}
Пусть $[a, b] \subset \mathbb{R}$ и функция $f$ монотонна на $[a, b]$. Доказать, что $f$ измерима относительно классической меры Лебега на $[a, b]$.
\end{task}

\begin{solution}
Пусть для определенности $f(x)$ --- невозрастающая на $[a, b]$ функция. Тогда для всех $c \in \mathbb{R}$ множество $f^{-1}((c, +\infty])$ является либо полуинтервалом $[a, d), d \in (a, b]$, либо отрезком $[a, d], d \in [a, b]$, либо пустым. Следовательно, оно измеримо, а значит $f(x)$ измерима.
\end{solution}
\begin{task}{8}
Построить такую функцию $f(x) \in C([0, 1])$, что для некоторого множества $A \subset [0, 1]$ меры нуль множество $f(A)$ измеримо и $\mu(f(A)) > 0$, где $\mu$ --- классическая мера Лебега.
\end{task}

\begin{solution}
Рассмотрим функцию $f(x) = \dfrac{1}{2}(x + \varphi(x))$, где $\varphi(x)$ --- канторова лестница. Функция $f(x)$ непрерывна и монотонна, $f([0, 1]) = [0, 1]$. Пусть также $P_0$ --- канторово множество, $\mu(P_0) = 0$. Тогда
$$ G = [0, 1] \setminus P_0 = \bigsqcup_{n=1}^{\infty} (a_n, b_n).$$
Для каждого интервала $\mu(f((a_n, b_n))) = \dfrac{1}{2}(b_n - a_n)$, и т.к $\mu(G) = 1$, мера множества $f(G)$ равна $\dfrac{1}{2}$. Тогда $\mu(f(P_0)) = \mu(f([0, 1])) - \mu(f(G)) = 1 - \dfrac{1}{2} = \dfrac{1}{2}$.
\end{solution}
\begin{task}{9}
Построить такую строго монотонную функцию $f(x) \in C([0, 1])$, что для некоторого множества $A \subset [0, 1]$ меры нуль множество $f(A)$ измеримо и $\mu(f(A)) > 0$, где $\mu$ --- классическая мера Лебега.
\end{task}

\begin{solution}
См. предыдущую задачу.
\end{solution}
\begin{task}{10}
Построить функцию $f(x) \in C([0, 1])$ и измеримое множество $A \subset \mathbb{R}$, для которых множество $f^{-1}(A)$ неизмеримо относительно классической меры Лебега.
\end{task}

\begin{solution}
Рассмотрим функцию $\psi(x) = \frac{1}{2}(x + \varphi(x))$, где $\varphi(x)$ --- канторова лестница. Функция $\psi(x)$ непрерывна и монотонна, следовательно, существует непрерывная функция $f(y) = \psi^{-1}(y): [0, 1] \rightarrow [0, 1]$. Пусть также $P_0$ --- канторово множество. В $\psi(P_0)$ существует неизмеримое подмножество, обозначим его $B$. Пусть теперь $A = f(B) = \psi^{-1}(B)$. Тогда $A \subset P_0$, следовательно $A \in \mathfrak{M}$ и $\mu(A) = 0$ в силу полноты меры Лебега, но $f^{-1}(A) = B \notin \mathfrak{M}$.
\end{solution}
\begin{task}{11}
Построить такую функцию $g(x) \in C([0, 1])$, что для некоторого измеримого множество $A \subset [0,1]$ c $\mu(A) = 0$, для которых множество $g(A)$ неизмеримо относительно классической меры Лебега.
\end{task}

\begin{solution}
Пусть $A$ --- множество, построенное в предыдущей задаче. Возьмем $g(x) = \psi(x)$. Тогда $g(A) = B \notin \mathfrak{M}$.
\end{solution}
\begin{task}{12}
Построить множество $A \subset [0, 1]$, которое измеримо относительно классической меры Лебега, но не является борелевским.
\end{task}

\begin{solution}
Пусть $A$ --- множество, построенное в двух предыдущих задачах. Предположим, что $A$ борелевское. Тогда, т.к $f$ измерима, то измеримо множество $f^{-1}(A)$. Однако $f^{-1}(A) = B \notin \mathfrak{M}$. Следовательно, $A$ не является борелевским множеством.
\end{solution}

%\section{Сходимость.}
%\begin{task}{2}
    Показать, что вообще говоря из сходимости п.в. не следует сходимость по мере в случае, когда мера $\sigma$-конечна.
    
\end{task}
\begin{solution}
    Рассмотрим $f_n(x) = \mathbf{I}_{[-n; n]}$ на $\mathbb{R}$. Тогда при $n \rightarrow \infty \hookrightarrow \mu(\{x~|~ f_n(x) \nrightarrow 1\}) = \{\varnothing\} = 0$\hspace{2mm} (так как $ ~ \forall x ~ \exists  n = \lceil x \rceil \hookrightarrow f_n(x) = 1$). Но при этом $\mu(\{x~|~ |f_n(x) - 1| > \frac{1}{2} \}) = \mu((- \infty;n) \cap (n;+\infty)) \neq 0$. 
\end{solution}
%\begin{task}{4}
Пусть  $\mathbb{Q}_[0, 1] = \{r_n\}_{n=1}^{\infty}$,Доказать, что последовательность 
$$f_n(x) = 
\begin{cases}
0, \text{ если } x=r_n \\
\frac{1}{\sqrt{n}(x - r_n}, \text{ если } x \in [0, 1] \setminus \{r_n\}
\end{cases}$$ сходится по классической мере Лебега на $[0, 1]$.
\end{task}

\begin{solution}
Пусть дано $\varepsilon > 0$. Тогда
$$E_n = \{ x \in [0, 1]: f_n(x) > \varepsilon\} = \left( \left(r_n - \dfrac{1}{\sqrt{n}\varepsilon}, r_n - \dfrac{1}{\sqrt{n}\varepsilon} \right) \setminus {r_n} \right) \cap [0, 1] $$
для $n \in \mathbb{N}$. Тогда
$$ \mu(E_n) \leqslant \dfrac{2}{\sqrt{n}\varepsilon} \to 0 \text{ при } n \to \infty $$
Следовательно, $f_n$ сходится по классической мере Лебега на $[0, 1]$.
\end{solution}
%\begin{task}{5}
Пусть  $\mathbb{Q}_[0, 1] = \{r_n = \frac{p_n}{q_n}\}_{n=1}^{\infty}$, где $p_n, q_n$ --- взаимно простые натуральные числа, $n \in \mathbb{N}$. Доказать, что последовательность $\{f_n(x)\}_{n=1}^{\infty}$, где $f_n(x) = e^{(-p_n - q_nx)^2}$ сходится по классической мере Лебега на $[0, 1]$, но не сходится ни в одной точке.
\end{task}

\begin{solution}
Пусть $\delta \in (0, \frac{1}{2})$. Если $x \in [0, 1] \setminus (r_n - \delta, r_n + \delta)$, то $f_n(x) \leqslant e^{-q_n^2\delta^2} \to 0$ при $n \to \infty$. Поэтому для любого $\varepsilon > 0$ существует такое $N$, что при $n > N$ справедлива оценка:
$$\mu(\{x \in [0, 1]: f_n(x) > \varepsilon\}) \leqslant \mu(r_n - \delta, r_n + \delta) = 2\delta $$, т.е $\{f_n\}$ сходится по мере к $0$ на $[0, 1]$.

Покажем, что $f_n(x)$ не сходится ни в одной точке. Из того, что любое действительное число можно со сколь угодно большой точностью приблизить рациональным, следует, что  для любого простого $q$ существует $r_m = \dfrac{p_m}{q}: 0 < |r_m - x_0| \leqslant \dfrac{1}{q}$. Тогда
$$ f_m(x_0) = e^{-q^2(r_m - x_0}^2 \geqslant e^{-1}.$$
Таким образом, существует подпоследовательность, не сходящаяся к нулю, следовательно $f_n(x)$ не сходится к $0$ поточечно. С другой стороны $f_n(x)$ не может сходиться к $f(x)$, отличной от нуля, т.к это противоречит сходимости к нулю по мере. 
\end{solution}
%\begin{task}{6}
Показать, что утверждение теоремы Егорова не выполняется для классической меры Лебега.
\end{task}

\begin{solution}
\textbf{Теорема Егорова}: Пусть $\mu(A) < \infty, f_n(x) \to f(x)$ п.в на $A$. Тогда $\forall \varepsilon > 0$ существует такое измеримое множество $E_{\varepsilon} \subseteq A$, что $\mu(A \setminus E_{\varepsilon}) < \varepsilon$ и последовательность $\{f_n(x)\}$ равномерно сходится на $E$.

Пусть $f_n(x) = \mathbb{I}_{[-n, n]}(x)$. В силу задачи 2 $f(x) \to 1$, однако $\mu(\{x \in \mathbb{R}: |f_n(x) - 1| < \frac{1}{2}\}) = \infty$ при всех $n$. Следовательно, условие теоремы Егорова не выполняется.
\end{solution}
%\section{Интеграл Лебега.}
%\begin{task}{1}
Поймите, что функция $f(x) = \mathbb{I}(x \in \mathbb{Q})$ интегрируема на $\mathbb{R}$, найдите величину интеграла.
\end{task}

\begin{solution} 
%Простая функция $h(x) = \sum_{i=1}^n c_i\mathbb{I}(x \in \mathbb{Q})$.\newline
%По определению интегралом Лебега простой функции $h(x)= \sum_{i=1}^n c_i\mathbb{I}(x \in \mathbb{Q})$ на множетсве $\mathbb{R}$ называется величина:
%\[
%\int_\mathbb{R} f(x)d\mu = \sum_{i = 1}^nc_i\mu()
%\]
0
\end{solution}
%\begin{task}{2}
Пусть $f(x) = \dfrac{1}{x}$ и $\mu$ — классическая мера Лебега на $(0; 1)$. Доказать, что
\[
\int\limits_0^1 f(x) d\mu = \infty,
\]
используя только определение интеграла Лебега.
\end{task}

\begin{solution} 

\end{solution}
%\begin{task}{3}
Верно ли, что функция $f(x) = \dfrac{\sin{x}}{x}$ интегрируема по Лебегу на прямой?
\end{task}

\begin{solution} 
\begin{align*}
    &f(x) = \dfrac{\sin{x}}{x} \\
    &\int\limits_{\pi (n-1)}^{\pi n}\dfrac{|\sin{x}|}{x}dx \geqslant \dfrac{1}{\pi n}\int\limits_{\pi (n-1)}^{\pi n}|\sin{x}|dx = \dfrac{2}{\pi n}.\\
    &\int\limits_0^{\pi N}\dfrac{|\sin{x}|}{x}dx \geqslant \dfrac{2}{\pi} \sum_{n = 1}^N\dfrac{1}{n} \text{ -- ряд расходится, значит}\\
    &\int\limits_0^\infty \dfrac{|\sin{x}|}{x}dx = +\infty \text{ т.е. не интегрируема по лебегу}
\end{align*}
\end{solution}
%\begin{task}{4}
Пусть $f(x)$ интегрируема по Лебегу на множестве $A$ (т.е. $\int_A|f(x)|d\mu < \infty$, будем писать $f \in L_1(A)$). Доказать, что $\mu(\lbrace x \in A : f(x) = \pm \infty \rbrace) = 0$.
\end{task}

\begin{solution} 
Можно считать, что $f(x) \geqslant 0$ на $A$. Пусть $A_1 = \lbrace x \in A : f(x) = +\infty\rbrace \in M$. Предположим, что $\mu(A_1) > 0$. Положим $A_2 = A_1$, если $\mu(A_1) < \infty$, иначе выберем множество $A_2 \subset A_1, A_2 \in M$ с $0 < \mu(A_2) < \infty$. Определим простые функции $h_n(x) = n\chi_{A_2}(x)$ для $n \in \mathbb(N)$. Ясно, что $0 \leqslant h_n(x) \leqslant f(x)$ при $n \in \mathbb{N}$ и $x \in E$, т.е. $h_n(x) \in Q_f$. Тогда по определению интеграла Лебега получаем, что 
\[
\int_A f(x)d\mu \geqslant \sup_n \int_A h_n(x)d\mu = \sup_n n\mu(A_2) = \infty,
\]
а это противоречит условию интегрируемости $f$.
\end{solution}
%\begin{task}{5}
Построить такую последовательность $\lbrace f_n(x)\rbrace_{n=1}^\infty$ неотрицательных функций из $L_1([0;1])$, таких что $f_n(x) \rightarrow 0$ при $n \rightarrow \infty$ для каждого $x \in [0; 1]$, но
\[
\int\limits_0^1 f_n(x)d\mu \nrightarrow 0, n \rightarrow \infty.
\]
\end{task}

\begin{solution} 
Пусть $f_n(x) = n\chi_{(0,\frac{1}{n})}(x)$ при $x \in [0, 1]$ и $n \in \mathbb{N}$. Тогда $f_n(x) \xrightarrow{n \rightarrow \infty} 0$ для каждого $x \in [0, 1]$, но 
\[
\int\limits_{[0,1]}f_n(x)d\mu = 1
\]
при всех $n$.
\end{solution}
\end{document}
